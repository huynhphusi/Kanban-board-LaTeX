\documentclass[12pt,a4paper,landscape]{book}
\usepackage[top=1.2cm, headsep=0.5cm, bottom=1cm, footskip=0.7cm, outer=1cm, inner=2cm] {geometry}
\usepackage{amsmath,amssymb,enumerate,enumitem,mathrsfs,multicol,systeme}
\usepackage{framed}
\usepackage[most]{tcolorbox}
\usepackage{fontawesome}
\usepackage{tikz,tkz-tab,tkz-euclide}
\usetikzlibrary{calc, angles, quotes, intersections, positioning, patterns, shadows, fit, decorations.text}
\usepackage{pgfplots}
\usepackage[utf8]{vietnam}
\usepackage[condensed,math]{anttor}
\usepackage{titlesec,titletoc}

\DeclareSymbolFont{symbolsC}{U}{txsyc}{m}{n}
\DeclareMathSymbol{\varparallel}{\mathrel}{symbolsC}{9}
\DeclareMathSymbol{\parallell}{\mathrel}{symbolsC}{9}
\DeclareMathSymbol{\nparallell}{\mathrel}{symbolsC}{11}
\renewcommand{\parallel}{\!\parallell\!}
\everymath{\displaystyle}
%Vẽ cung tròn trên đầu
\usepackage{tipa}
\newcommand{\arc}[1]{{%
		\setbox9=\hbox{#1}%
		\ooalign{\resizebox{\wd9}{\height}{\texttoptiebar{\phantom{A}}}\cr#1}}}
%Kí hiệu tam giác đồng dạng
\newcommand{\similar}
{%
	\mathrel%
	{%
		\begin{tikzpicture}[baseline=(a.south),line width=0.15ex]
			\node (a) {};
			\draw (0,0.5ex) arc (90:270:0.5ex) .. controls +(0:0.5ex) and +(180:0.5ex) .. (1ex,0.5ex) arc (90:-90:0.5ex);
		\end{tikzpicture}
	}
}

%Vẽ kí hiệu vuông góc
\def\kihieuvuong[size=#1](#2,#3,#4){
	\draw[gray] ($(#3)!#1!(#2)$)--($($(#3)!#1!(#2)$)+($(#3)!#1!(#4)$)-(#3)$)--($(#3)!#1!(#4)$);
}

\definecolor{trello_bg_green}{HTML}{023020}
\definecolor{trello_bg_blue}{HTML}{191970}
\definecolor{trello_bg_pink}{HTML}{FF10F0}
\definecolor{trello_bg_violet}{HTML}{7F00FF}
\definecolor{trello_bg_orange}{HTML}{FF5733}

\definecolor{trello_bd_green}{HTML}{023020}
\definecolor{trello_bd_blue}{HTML}{191970}
\definecolor{trello_bd_pink}{HTML}{FF10F0}
\definecolor{trello_bd_violet}{HTML}{7F00FF}
\definecolor{trello_bd_orange}{HTML}{FF5733}

\newcounter{khung_trello}
\newtcolorbox[use counter=khung_trello]{khung_trello}[2]{size=title,
	shadow={0.2mm}{-0.3mm}{0mm}{black!40}, breakable, enhanced,
	%colback=red!5!white,
	colback=trello_bg_#2,
	colbacktitle=trello_bg_#2,titlerule=-1mm,
	colframe=trello_bd_#2,
	title={\vspace*{2mm}\hspace*{1mm}\bf\color{white}\large #1}, 
	sharp corners,rounded corners, arc=0.5mm,left=0mm,right=0mm,top=1mm,bottom=1mm
}
\newcounter{khung_trello_box}
\newtcolorbox[use counter=khung_trello_box]{khung_trello_box}[1][]{enhanced,
	%drop fuzzy shadow south,
	shadow={0mm}{-0.2mm}{0mm}{black!40},
	colframe=black,colback=white,boxrule=0pt,enlarge bottom by=-1mm,
	left=1mm,right=1mm,top=1mm,bottom=0mm
}

\begin{document}
	\pagestyle{fancy}
	\lhead{\fancyplain{}{\rightmark}}
	\chead{}
	\rhead{Tịch Tà Toán Phổ}
	%-------------------------------------
	\lfoot{\faMortarBoard{} \LaTeX: Huỳnh Phú Sĩ}
	\cfoot{\thepage}
	\rfoot{\faBank{ } Trường THCS-THPT Mỹ Thuận}
	%-------------------------------------
	\renewcommand{\headrulewidth}{0.4pt}
	\renewcommand{\footrulewidth}{0.4pt}
	\lhead{\fancyplain{}{\color{red}\bf NGUYÊN HÀM, TÍCH PHÂN, ỨNG DỤNG CỦA TÍCH PHÂN}}
	\begin{multicols*}{3}
		\begin{khung_trello}{Nguyên hàm}{green}
			\begin{khung_trello_box}
				\textbf{\color{trello_bg_green}Tính chất}
				\begin{itemize}[leftmargin=5mm]
					\item $\displaystyle\int f'(x)\mathrm{\,d}x=f(x)+C$.
					\item $\displaystyle\int k\cdot f(x)\mathrm{\,d}x=k\cdot\displaystyle\int f(x)\mathrm{\,d}x$ ($k\neq0$).
					%\item $\displaystyle\int\big[f(x)\pm g(x)\big]\mathrm{\,d}x=\int f(x)\mathrm{\,d}x\pm\int g(x)\mathrm{\,d}x$.
				\end{itemize}
			\end{khung_trello_box}
			\begin{khung_trello_box}
				\textbf{\color{trello_bg_green}Bảng nguyên hàm}
				\begin{itemize}[leftmargin=5mm]
					\item $\displaystyle\int0\mathrm{\,d}x=C$
					\item $\displaystyle\int\mathrm{\,d}x=x+C$
					\item $\displaystyle\int x^n\mathrm{\,d}x=\dfrac{x^{n+1}}{n+1}+C$ ($n\neq-1$)
					\item $\displaystyle\int\dfrac{1}{x}\mathrm{\,d}x=\ln|x|+C$
					\item $\displaystyle\int\mathrm{e}^x\mathrm{\,d}x=\mathrm{e}^x+C$
					\item $\displaystyle\int a^x\mathrm{\,d}x=\dfrac{a^x}{\ln a}+C$ ($0<a\neq1$)
					\item $\displaystyle\int\cos x\mathrm{\,d}x=\sin x+C$
					\item $\displaystyle\int\sin x\mathrm{\,d}x=-\cos x+C$
					\item $\displaystyle\int\dfrac{1}{\cos^2x}\mathrm{\,d}x=\tan x+C$
					\item $\displaystyle\int\dfrac{1}{\sin^2x}\mathrm{\,d}x=-\cot x+C$
				\end{itemize}
			\end{khung_trello_box}
		\end{khung_trello}
		\begin{khung_trello}{Tích phân}{orange}
			\begin{khung_trello_box}
				\textbf{\color{trello_bg_orange}Định nghĩa}
				$$\displaystyle\int\limits_{a}^{b}f(x)\mathrm{\,d}x=F(x)\bigg|_a^b=F(b)-F(a)$$
			\end{khung_trello_box}
			\begin{khung_trello_box}
				\textbf{\color{trello_bg_orange}Quy ước}
				\begin{itemize}[leftmargin=5mm]
					\item $\displaystyle\int\limits_{a}^{a}f(x)\mathrm{\,d}x=0$
					\item $\displaystyle\int\limits_{a}^{b}f(x)\mathrm{\,d}x=-\int\limits_{b}^{a}f(x)\mathrm{\,d}x$
				\end{itemize}
			\end{khung_trello_box}
			\begin{khung_trello_box}
				\textbf{\color{trello_bg_orange}Tính chất}
				\begin{itemize}[leftmargin=5mm]
					\item $\displaystyle\int\limits_{a}^{b}f'(x)\mathrm{\,d}x=f(b)-f(a)$.
					\item $\displaystyle\int\limits_{a}^{b}k\cdot f(x)\mathrm{\,d}x=k\cdot\int\limits_{a}^{b}f(x)\mathrm{\,d}x$.
					\item $\displaystyle\int\limits_{a}^{c}f(x)\mathrm{\,d}x+\int\limits_{c}^{b}f(x)\mathrm{\,d}x=\int\limits_{a}^{b}f(x)\mathrm{\,d}x$.
				\end{itemize}
			\end{khung_trello_box}
			\begin{khung_trello_box}
				\textbf{\color{trello_bg_orange}Phương pháp tích phân từng phần}
				$$\displaystyle\int\limits_{a}^{b}u\cdot v'\mathrm{\,d}x=uv\bigg|_a^b-\int\limits_{a}^{b}u'\cdot v\mathrm{\,d}x$$
			\end{khung_trello_box}
		\end{khung_trello}
		\vfill\null
		\columnbreak
		\begin{khung_trello}{Diện tích hình phẳng}{blue}
			\begin{khung_trello_box}
				$$S=\displaystyle\int\limits_{a}^{b}\big|f(x)\big|\mathrm{\,d}x$$
			\end{khung_trello_box}
			\begin{khung_trello_box}
				$$S=\displaystyle\int\limits_{a}^{b}\big|f(x)-g(x)\big|\mathrm{\,d}x$$
			\end{khung_trello_box}
		\end{khung_trello}
		\begin{khung_trello}{Thể tích khối tròn xoay}{pink}
			\begin{khung_trello_box}
				$$V=\displaystyle\int\limits_{a}^{b}S(x)\mathrm{\,d}x$$
				\textit{$S(x)$ là diện tích thiết diện mặt cắt}
			\end{khung_trello_box}
			\begin{khung_trello_box}
				$$V=\pi\displaystyle\int\limits_{a}^{b}f^2(x)\mathrm{\,d}x$$
			\end{khung_trello_box}
			\begin{khung_trello_box}
				$$V=\pi\displaystyle\int\limits_{a}^{b}\big[f^2(x)-g^2(x)\big]\mathrm{\,d}x$$
			\end{khung_trello_box}
		\end{khung_trello}
		\begin{khung_trello}{Quãng đường}{violet}
			\begin{khung_trello_box}
				$$s=\displaystyle\int\limits_{a}^{b}v(t)\mathrm{\,d}t$$
				\begin{itemize}[leftmargin=5mm]
					\item Lúc bắt đầu tính giờ: $t=0$
					\item Lúc vật dừng hẵn: $v(t)=0$
				\end{itemize}
			\end{khung_trello_box}
		\end{khung_trello}
	\end{multicols*}

	\newpage
	\lhead{\fancyplain{}{\color{red}\bf PHƯƠNG PHÁP TỌA ĐỘ TRONG KHÔNG GIAN}}
	\begin{multicols*}{3}
		\begin{khung_trello}{Hệ tọa độ $\mathbf{Oxyz}$}{blue}
			\begin{khung_trello_box}
				\textbf{\color{trello_bg_blue}Hệ trục tọa độ $\mathbf{Oxyz}$}
				\begin{itemize}[leftmargin=5mm,label={\color{trello_bg_blue}$\blacksquare$}]
					\item Gốc tọa độ: $O(0;0;0)$;
					\item Vectơ đơn vị: $\overrightarrow{i}\in Ox$, $\overrightarrow{j}\in Oy$, $\overrightarrow{k}\in Oz$;
					\item Trục tọa độ: $Ox$, $Oy$, $Oz$;
					\item Mặt phẳng tọa độ: $(Oxy)$, $(Oyz)$, $(Oxz)$.
				\end{itemize}
			\end{khung_trello_box}
			\begin{khung_trello_box}
				Điểm $M\big(x_0;y_0;z_0\big)$ nếu $$\overrightarrow{OM}=x_0\overrightarrow{i}+y_0\overrightarrow{j}+z_0\overrightarrow{k}$$
			\end{khung_trello_box}
			\begin{khung_trello_box}
				Cho hai vectơ $\overrightarrow{a}=\big(a_1;a_2;a_3\big)$, $\overrightarrow{b}=\big(b_1;b_2;b_3\big)$
				\begin{itemize}[leftmargin=5mm,label={\color{trello_bg_blue}$\heartsuit$}]
					\item $\overrightarrow{a}\pm\overrightarrow{b}=\big(a_1\pm b_1;a_2\pm b_2;a_3\pm b_3\big)$;
					\item $k\cdot\overrightarrow{a}=\big(k\cdot a_1;k\cdot a_2;k\cdot a_3\big)$;
					\item $\overrightarrow{a},\,\overrightarrow{b}$ cùng phương $\exists k$ sao cho $a_1=kb_1$, $a_2=kb_2$, $a_3=kb_3$.
				\end{itemize}
			\end{khung_trello_box}
			\begin{khung_trello_box}
				\begin{itemize}[leftmargin=5mm,label={\color{trello_bg_blue}$\heartsuit$}]
					\item $\overrightarrow{AB}=B-A=\big(x_B-x_A;y_B-y_A;z_B-z_A\big)$;
					\item Trung điểm của đoạn thẳng $AB$ là $$M=\dfrac{A+B}{2}$$
					\item Trọng tâm của tam giác $ABC$ là $$G=\dfrac{A+B+C}{3}$$
					\item $ABCD$ là hình bình hành: $A+C=B+D$
				\end{itemize}
			\end{khung_trello_box}
		\end{khung_trello}
		\begin{khung_trello}{Tích vô hướng}{violet}
			\begin{khung_trello_box}
				$$\overrightarrow{a}\cdot\overrightarrow{b}=a_1b_1+a_2b_2+a_3b_3$$
			\end{khung_trello_box}
			\begin{khung_trello_box}
				\textbf{\color{trello_bg_violet}Độ dài vectơ}
				$$\big|\overrightarrow{a}\big|=\sqrt{a_1^2+a_2^2+a_3^2}$$
			\end{khung_trello_box}
			\begin{khung_trello_box}
				\textbf{\color{trello_bg_violet}Góc giữa hai vectơ}
				$$\begin{aligned}
					\cos\big(\overrightarrow{a},\overrightarrow{b}\big)&=\dfrac{\overrightarrow{a}\cdot\overrightarrow{b}}{\big|\overrightarrow{a}\big|\cdot\big|\overrightarrow{b}\big|}\\
					&\dfrac{a_1b_1+a_2b_2+a_3b_3}{\sqrt{a_1^2+a_2^2+a_3^2}{\color{red}\cdot}\sqrt{b_1^2+b_2^2+b_3^2}}
				\end{aligned}$$
				\textit{Lưu ý:} $\overrightarrow{a}\perp\overrightarrow{b}\Leftrightarrow\overrightarrow{a}\cdot\overrightarrow{b}=0$.
			\end{khung_trello_box}
		\end{khung_trello}
		\begin{khung_trello}{Phương trình mặt cầu}{orange}
			\begin{khung_trello_box}
				Mặt cầu tâm $I(a;b;c)$, bán kính $R$ có phương trình
				$$(x-{\color{red}a})^2+(y-{\color{red}b})^2+(z-{\color{red}c})^2={\color{red}R}^2$$
			\end{khung_trello_box}
			\begin{khung_trello_box}
				$$x^2+y^2+z^2-2{\color{red}a}x-2{\color{red}b}y-2{\color{red}c}z+{\color{red}d}=0$$
				\begin{itemize}[leftmargin=5mm]
					\item Tâm $I(a;b;c)$;
					\item Bán kính $R=\sqrt{a^2+b^2+c^2-d}$\\
						  Điều kiện: $a^2+b^2+c^2-d>0$
				\end{itemize}
			\end{khung_trello_box}
		\end{khung_trello}
		\vfill\null
		\columnbreak
		\begin{khung_trello}{Phương trình mặt phẳng}{pink}
			\begin{khung_trello_box}
				\textbf{\color{trello_bg_pink}Vectơ pháp tuyến} của mặt phẳng là vectơ có \textbf{giá} vuông góc với mặt phẳng.
				\begin{itemize}[leftmargin=5mm]
					\item Mỗi mặt phẳng có vô số vectơ pháp tuyến;
					\item Nếu $\overrightarrow{n}=(a;b;c)$ là vectơ pháp tuyến của mặt phẳng $(\alpha)$ thì $k\overrightarrow{n}=(ka;kb;kc)$ cũng là vectơ pháp tuyến của $(\alpha)$.
				\end{itemize}
			\end{khung_trello_box}
			\begin{khung_trello_box}
				\textbf{\color{trello_bg_pink}Phương trình tổng quát của mặt phẳng}
				$${\color{red}a}\big(x-x_0\big)+{\color{red}b}\big(y-y_0\big)+{\color{red}c}\big(z-z_0\big)=0$$
				\begin{itemize}[leftmargin=5mm]
					\item $\overrightarrow{n}=({\color{red}a};{\color{red}b};{\color{red}c})$ là vectơ pháp tuyến;
					\item $M\big(x_0;y_0;z_0\big)$ là điểm thuộc mặt phẳng.
				\end{itemize}
			\end{khung_trello_box}
			\begin{khung_trello_box}
				\textbf{\color{trello_bg_pink}Phương trình đoạn chắn}\\
				Nếu mặt phẳng $(\alpha)$ cắt $Ox$ tại $A({\color{red}a};0;0)$, cắt $Oy$ tại $B(0;{\color{red}b};0)$ cắt $Oz$ tại $C(0;0;{\color{red}c})$ thì
				$$(\alpha)\colon\dfrac{x}{\color{red}a}+\dfrac{y}{\color{red}b}+\dfrac{z}{\color{red}c}=1$$
			\end{khung_trello_box}
		\end{khung_trello}
		\begin{khung_trello}{Góc và khoảng cách}{green}
			\begin{khung_trello_box}
				\textbf{\color{trello_bg_green}Khoảng cách từ một điểm đến mặt phẳng}
				%Khoảng cách từ điểm $M\big(x_0;y_0;z_0\big)$ đến mặt phẳng $(\alpha)\colon{\color{red}a}x+{\color{red}b}y+{\color{red}c}z+d=0$ là
				$$\mathrm{d}\big(M,(\alpha)\big)=\dfrac{{\color{red}\big|}{\color{red}a}x_0+{\color{red}b}y_0+{\color{red}c}z_0+d{\color{red}\big|}}{\sqrt{{\color{red}a}^2+{\color{red}b}^2+{\color{red}c}^2}}$$
			\end{khung_trello_box}
			\begin{khung_trello_box}
				\textbf{\color{trello_bg_green}Góc giữa hai mặt phẳng}
				$$\cos\big((\alpha),(\beta)\big)=\dfrac{{\color{red}\big|}a_1a_2+b_1b_2+c_1c_2{\color{red}\big|}}{\sqrt{a_1^2+b_1^2+c_1^2}{\color{red}\cdot}\sqrt{a_2^2+b_2^2+c_2^2}}$$
			\end{khung_trello_box}
		\end{khung_trello}
	\end{multicols*}
\end{document}
